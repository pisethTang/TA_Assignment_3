% LaTeX conversion of the provided page (content unchanged)
\documentclass[12pt]{article}
\usepackage{amsmath,amssymb}
\DeclareMathOperator*{\argmax}{arg\,max}
\DeclareMathOperator*{\argmin}{arg\,min}
\usepackage[margin=1in]{geometry}
\usepackage{enumitem}
\newcommand{\R}{\rm I\!R}
\begin{document}
\paragraph{Functions}
\begin{quote}
    The \emph{coverage} problem is a problem relates to \emph{graph
    theory}. We can define the \emph{submodular and monotone} using 
    the notation from graph theory. Given a graph $G=(V,E)$. Our 
    search space is the vertex space $V$. The submodular function
    can be defined as:
    \[
        f: 2^V\rightarrow\R
    \]
    $A\subseteq B\subseteq 2^V$ are sets of vertices, it is 
    monotonic when:
    \[
    A\subseteq B\implies f(A)\le f(B).
    \]
    it is \emph{submodular}, given that $v\in V\setminus B$ when:
    \[
    f(B\cup\{v\})-f(B)\le f(A\cup\{v\})-f(A).
    \]
\end{quote}
\bigskip
\paragraph{Objective Functions}
\begin{quote}
    I will defined $\boldsymbol{v}\subseteq V$ be a set of vertices, the relevant notation will be written in bold with a lowercase letter. The bold, lowercase letter indicate a set of values.
\end{quote}
\textbf{Uniform $k$-constraint cost}
\begin{quote}
    The \emph{uniform cost} function $c:2^V\rightarrow\R_{\geq0}$ 
    is a linear function
    with weight $\boldsymbol{w}$ and defined as:
    \begin{equation*}
        \begin{aligned}
            c(\boldsymbol{v}) = \sum_i^n w_i\cdot c(v_i)\leq k\in\R,\\
            \text{where } w_i = 1,\;c(v_i) = 1
        \end{aligned}
    \end{equation*}
    Notice that $c(v_i)$ is the cost of the vertex being chosen in a
    set not the cost of its out-going edge. The value for 
    $\boldsymbol{w} = \boldsymbol{1}$ and $c(v_i) = 1$ is given.
\end{quote}
\textbf{Fitness Function}
\begin{quote}
    For the \emph{Maximum Coverage}
    \[
        \argmax_{\boldsymbol{v}\subseteq V} f(\boldsymbol{v}) = |N(\boldsymbol{v})|,\quad\text{ where }
        N(\boldsymbol{v})\text{ is the out-going neighbours of } 
        \boldsymbol{v}
    \]
    For the \emph{Max Influence}, this is a probabilistic problem,
    therefore we need to find the \emph{seed set} 
    $\boldsymbol{v}\in V$ that has the highest expected influence 
    value.
    \[
        \argmax_{\boldsymbol{v}\subseteq V} f(\boldsymbol{v}) = 
        \mathbb{E}[IC(\boldsymbol{v})]
    \]
    \begin{quote}
        $IC(.)$ is a \emph{Independence Cascade process}, in short, 
        this is a propagation process that how a vertex $v$ can 
        influence vertex $u$ with probability $p_{v,u}$ from time
        $t\rightarrow t+1$.
    \end{quote}
\end{quote}
\begin{quote}
    Overall, the fitness of a search point is a 2D vector given by.
    Therefore this is a bi-objective optimisation problem, defined
    as
    \begin{equation*}
        \begin{aligned}
            &\argmax_{\boldsymbol{v}\subseteq V}\mathsf{F}(\boldsymbol{v}) 
            = (f^\prime(\boldsymbol{v}),-c(\boldsymbol{v}))\\
            &\qquad\quad\text{subject to}\\
            &\qquad\qquad c(\boldsymbol{v})\leq 10\\
            &\qquad\qquad f^\prime(\boldsymbol{v})=\begin{cases}
                f(\boldsymbol{v}),&\quad c(\boldsymbol{v})\leq k=10\\
                (k = 10) - f(\boldsymbol{v})&,\quad c(\boldsymbol{v})> k=10
            \end{cases}
        \end{aligned} 
    \end{equation*}
\end{quote}
\bigskip
\paragraph{Dominance Formulation}
\begin{quote}
    For any two given solution sets 
    $\boldsymbol{v}\succ\boldsymbol{u}$. We have that
    \begin{equation*}
        \begin{aligned}
            &\mathsf{F}(\boldsymbol{v})\succ\mathsf{F}
            (\boldsymbol{u})\\
            &\iff\begin{cases}
                f^\prime(\boldsymbol{v})\geq f^\prime(\boldsymbol{u})\\
                c(\boldsymbol{v})\leq c(\boldsymbol{u})\\
                \mathsf{F}(\boldsymbol{v})\neq\mathsf{F}
                (\boldsymbol{u})
            \end{cases}
        \end{aligned}
    \end{equation*}
    Thus, we can say that
    \begin{equation*}
        \begin{aligned}
            \mathsf{F}(\boldsymbol{v})\succ\mathsf{F}
            (\boldsymbol{u})\iff\begin{cases}
                f^\prime(\boldsymbol{v})\geq f^\prime(\boldsymbol{u})\\
                c(\boldsymbol{v})< c(\boldsymbol{u})\\
            \end{cases}
        \end{aligned}
    \end{equation*}
    And also,
    \begin{equation*}
        \begin{aligned}
            \mathsf{F}(\boldsymbol{v})\succeq\mathsf{F}
            (\boldsymbol{u})\iff\begin{cases}
                 f^\prime(\boldsymbol{v})\geq f^\prime(\boldsymbol{u})\\
                c(\boldsymbol{v})\leq c(\boldsymbol{u})\\
            \end{cases}
        \end{aligned}
    \end{equation*}

\end{quote}
\end{document}
