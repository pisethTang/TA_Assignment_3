\documentclass[a4paper, 11pt]{article}
\usepackage[margin=2cm]{geometry}
\usepackage{amsmath, amssymb, amsfonts, xcolor}
\usepackage{hyperref}
\usepackage[backend=biber,style=authoryear]{biblatex}
\addbibresource{refs.bib}

\hypersetup{
    colorlinks=true,
    citecolor=teal,
    linkcolor=blue,
    urlcolor=magenta,
}
\newcommand{\R}{{\rm I\!R}}

\begin{document}
\section{Prerequisites}
\subsection{Ordered Set}
\paragraph{Partial Ordered Set}
    \begin{quote}
        Formally, a partial order is a \textbf{homogeneous binary relation} that is \textbf{reflexive}, \textbf{antisymmetric}, and \textbf{transitive}. 
        A \textbf{partially ordered set} (\textit{poset} for short) is an \textit{ordered pair} $P = (X, \le)$ consisting of a set $X$ (called the 
        \textit{ground set} of $P$) and a partial order $\le$ on $X$. When the meaning is clear from context and there is no ambiguity about the partial 
        order, the set $X$ itself is sometimes called a poset.

        A \textbf{reflexive, weak,} or \textbf{non-strict partial order}, commonly referred to simply as a \textbf{partial order}, is a \textbf{homogeneous relation} $\le$ on a set $P$ that is \textbf{reflexive}, \textbf{antisymmetric}, and \textbf{transitive}. That is, for all $a, b, c \in P$, it must satisfy:
        \begin{enumerate}
            \item \textbf{Reflexivity:} $a \le a$, i.e. every element is related to itself.
            \item \textbf{Antisymmetry:} if $a \le b$ and $b \le a$ then $a = b$, i.e. no two distinct elements precede each other.
            \item \textbf{Transitivity:} if $a \le b$ and $b \le c$ then $a \le c$.
        \end{enumerate}
        A non-strict partial order is also known as an \textbf{antisymmetric preorder}.
    \end{quote}
\paragraph{Total Order}
    \begin{quote}
        In mathematics, a total order or linear order is a partial order \emph{\textbf{in which any two elements are comparable}}. That 
        is, a total order is a binary relation $\le$ on some set $X$, which satisfies the following for all $a, b$ and $c$ in $X$:
        \begin{enumerate}
            \item $a \le a$ (\textit{reflexive}).
            \item If $a \le b$ and $b \le c$ then $a \le c$ (\textit{transitive}).
            \item If $a \le b$ and $b \le a$ then $a = b$ (\textit{antisymmetric}).
            \item $a \le b$ or $b \le a$ (\textit{strongly connected}, formerly called \textit{totality}).
        \end{enumerate}
    \end{quote}
\subsection{Cartesian Product}
    \begin{quote}
        In \textbf{mathematics}, specifically \textbf{set theory}, the \textbf{Cartesian product} of two sets $A$ and $B$, denoted $A \times B$, is the set 
        of all \textbf{ordered pairs} $(a, b)$ where $a$ is an element of $A$ and $b$ is an element of $B$. In terms of \textbf{set-builder notation}, that is
        \[
            A \times B = \{(a, b) \mid a \in A \ \text{and} \ b \in B\}.
        \]
    \end{quote}
    \paragraph{Orders on the Cartesian product of totally ordered sets}
        \begin{quote}
            There are several ways to take two totally ordered sets and extend to an order on the \textbf{Cartesian product}, though the resulting order may 
            only be \textbf{partial}. Here are three of these possible orders, listed such that each order is stronger than the next:
            \begin{itemize}
                \item \textbf{Lexicographical order:} $(a,b) \le (c,d)$ if and only if $a < c$ or $(a = c \text{ and } b \le d)$. This is a total order.
                \item $(a,b) \le (c,d)$ if and only if $a \le c$ and $b \le d$ (the \textbf{product order}). This is a partial order.
                \item $(a,b) \le (c,d)$ if and only if $(a < c \text{ and } b < d)$ or $(a = c \text{ and } b = d)$ (the reflexive closure of the \textbf{direct product} of the corresponding strict total orders). This is also a partial order.
            \end{itemize}
        \end{quote}
    \paragraph{Relations as Subsets of Cartesian Products}\href{https://math.stackexchange.com/questions/2454926/relations-cartesian-product-explanation}{Mathematics Exchange}
        \begin{quote}
            The set of ordered pairs drawn from the Cartesian product can be defined as
            \begin{equation*}
                R = \left\{(x,y)\in A\times B|\;xRy\right\}
            \end{equation*}
            Thus we can say
            \begin{equation*}
                R\subseteq A\times B
            \end{equation*}
        \end{quote}
\subsection{Pareto Set}
    \begin{quote}
        We assume w.l.o.g that $k$ objective functions,
        \begin{equation*}
            \begin{aligned}
                f_i&:X^k\rightarrow\R\\
                f(x) &= \bigl(f_1(x),f_2(x),\ldots,f_k(x)\bigr)\in\R^k,\quad\text{where }\mathbf{x}\in X^k
            \end{aligned} 
        \end{equation*}
        A solution $x\in X$  is said to \emph{dominate} another solution $y\in X$ iff $\forall\;1\leq i\leq k:f_i(x)\leq f_i(y)$ and $\exists\;1\leq i\leq k:f_i(x)<f_i(y)$. This can denoted as
        $x\prec y$. A solution $x\in X$ \emph{weakly dominates} a solution $y\in X$ iff $\forall\;1\leq i\leq k:f_i(x)\leq f_i(y)$. This can be denoted as
        $x\preceq y$.
    \end{quote}
    \paragraph{Parento Optimal}
        \begin{quote}
            A solution $x^*\in X$ is then called \emph{Parento Optimal} iff there is no other solution in $X$ that dominates $x^*$. 
        \end{quote}
    \paragraph{Pareto set approximations}
        \begin{quote}
            Specific sets of solutions are the so-called \emph{Pareto set approximations}, which are solution sets of pairwisely non-dominated solutions.
        \end{quote}
\section{Deteriorative Cycles}  
    \begin{quote}
        In single-objective optimization, every solution is mapped to a real value and solutions can always be pairwisely compared via the less or equal 
        relation $\leq$ on $\R$. In another words, the total order (any elements are comparable) $\leq\subseteq\R\times\R$ induces via $f$ order on the 
        search space $X$ that is a total preorder. In a multiobjective scenario, the $\le$ relation is generalized to objective vectors, i.e., $\le$ is 
        a subset of $\R^k\times\R^k$. Here, the totality is not given due to vectors $a, b \in \R^k$ where $f_1(a) < f_1(b)$ 
        but $f_2(a) > f_2(b)$—the relation $\le$ on the set of objective vectors is only a partial order, i.e., reflexive, antisymmetric, and transitive. 
        This means that \emph{not any elements are comparable}. \fbox{\cite{brockhoff2009theoretical}}
    \end{quote}
\subsection{Downfall of The Evolutionary Algorithm Multiobjective Optimisation}
    \begin{quote}
        Among the well-established ones, NSGA-II [Deb et al., 2002] and SPEA2 [Zitzler et al., 2002] have to be mentioned here. Both use the Pareto dominance 
        concept as the main selection criterion in an elitist manner where non-dominated solutions are favored over dominated ones. In addition, a second selection 
        criterion establishes diversity among the solutions. However, experimental studies have shown that both algorithms do not scale well if the number 
        of objectives increases and that a cyclic behavior can be observed. This means that—although non-dominated solutions are preferred over dominated 
        ones—over time, previously dominated solutions enter the population again, resulting in an oscillating distance to the Pareto front.
        \fbox{\cite{brockhoff2009theoretical}}
    \end{quote}

\newpage
\printbibliography
\end{document}